% --------------------------------------------------------------
% This is all preamble stuff that you don't have to worry about.
% Head down to where it says "Start here"
% --------------------------------------------------------------
 
\documentclass[12pt]{article}
 
\usepackage[margin=1in]{geometry} 
\usepackage{amsmath,amsthm,amssymb}
\setlength{\parskip}{1em}

\newcommand{\N}{\mathbb{N}}
\newcommand{\Z}{\mathbb{Z}}
 
\newenvironment{theorem}[2][Theorem]{\begin{trivlist}
\item [\hskip \labelsep {\bfseries #1}\hskip \labelsep {\bfseries #2.}]}{\end{trivlist}}
\newenvironment{lemma}[2][Lemma]{\begin{trivlist}
\item [\hskip \labelsep {\bfseries #1}\hskip \labelsep {\bfseries #2.}]}{\end{trivlist}}
\newenvironment{exercise}[2][Exercise]{\begin{trivlist}
\item [\hskip \labelsep {\bfseries #1}\hskip \labelsep {\bfseries #2.}]}{\end{trivlist}}
\newenvironment{problem}[2][Problem]{\begin{trivlist}
\item [\hskip \labelsep {\bfseries #1}\hskip \labelsep {\bfseries #2.}]}{\end{trivlist}}
\newenvironment{question}[2][Question]{\begin{trivlist}
\item [\hskip \labelsep {\bfseries #1}\hskip \labelsep {\bfseries #2.}]}{\end{trivlist}}
\newenvironment{corollary}[2][Corollary]{\begin{trivlist}
\item [\hskip \labelsep {\bfseries #1}\hskip \labelsep {\bfseries #2.}]}{\end{trivlist}}

\newenvironment{solution}{\begin{proof}[Solution]}{\end{proof}}
 
\begin{document}
 
% --------------------------------------------------------------
%                         Start here
% --------------------------------------------------------------
 
\title{Week 2 Writeup}
\author{Arthur Liou}

\maketitle

Prompt: Submitting a write-up of your thoughts, impressions, and any conclusions based on the material from the week. Each week will have its own assignment in the grades page.
\par
\linebreak
For this week’s writeup, I’m reflecting on the lectures content that we had over the two lessons (and multiple videos per lesson). We started in diving into forensic analysis, with the three steps for forensic analysis. Then we went deeper into the cycles, ensuring what to do and how to preserve evidence during it acquisition, and how this is important when dealing with legal ramifications. Handling of evidence is critical because of what law enforcement can discover and use in court to prove intent/malware/evidence/etc. Further along the lectures, we found that there are probably hundreds (if not millions!) of different sources from where we can collect and / or use as evidence.

I liked going through this week more since we went through a lot of different parts of forensic analysis and heard a lot more stories about real-world applications of this material. Also Day 2 was very informative. When I was young, I had a Windows PC and liked to tinker with software and settings, so seeing familiar actions/programs/features, such as regedit, really clicked and resonated with me as aspects of Windows PCs that I was previously familiar with, and now am / will be using to learn about cyberdefense!

\newpage
\newline
Lecture Notes
\newline
Advanced Forensics Lesson 1
\begin{itemize}
\item Intro to this week’s material and time split
\item Incident response
\item How to best react to incident response, Create a timeline, Analyzing memory, memory dump, book around Hacking – The Cuckoo’s Nest, 
\item Cases for using forensics: fraud, IP theft, inappropriate child behavior, eDiscovery supporting,  
\item As a forensic investigator this time around, we’re only proving what happened on the system. Specialty in an area, use the data to discover the facts, replicating and showing what happened
\item 3 Steps 
\item 1) Evidence acquisition
\item 2) Investigation and analysis,
\item 3) Reporting results
\item Minimize data loss, record everything, analyze all data collected (evidence), report findings
\item Story: Don’t pull the plug since also clean out memory too
\item Post-modern devices, like Xbox, and PlayStation
\item Always do a memory dump
\item Time – is very important
\item What is Evidence? Network, operating systems, databases, CD, USB, humans
\item Evidence handling – preserve integrity of the evidence at all times
\item The IR Process – Incident Occurs, Take Action, Remediation, Evaluation
\item Legal Topics for IR & Forensic Analysts. When dealing with digital evidence, ensuring that you have access and gather all the available evidence is paramount.
\item Mapping Evidence to an APT-Case - Recon, Weaponization, Deliver, Exploitation, Installation, Command and Control, Actions on Objectives
\item Investigation Cycle – Verification, System Description, Evidence Acquisition. Then Cycle – Reporting Analysis, Timeline Analysis, Media Analysis, String or Byte Search, Data Recovery
\item What to Acquire: Memory (virtual and physical) Drive, physical and logical (entire drive and partition), network traffic )full packet captures)
\item Locard’s Exchange Principle states that when any two object come into contact, there is always transference of material from each object onto the other. You can’t interact with a live system without having some effect on it.
\item Evidence: “Once Contaminated – stay contaminated = compromised evidence”
\item Initial Response: Pull the plug or turn the machine off
\item Order of volatility – when collecting evidence you should proceed from the volatile to the less volatile (see RFC 3227). 
\item Example order of volatility: System Memory, Temporary File Systems (swapfile / paging file), Process Table & Network Connections, Specific Process Information May Be Dumped, Network Routing Information & ARP Cache, Forensics Acquisition of Disks, Remote Logging & Monitoring Data, Physical configuration & network topology, Backups
\item Live Response on Windows: Obtain volatile data (which would be lost upon shutdown), obtain the non-volatile data (time stamps, event logs, web logs, registry), Obtain any relevant, logical files – unknown executables, attacker tools
\item Lab 1 – Hands On Evidence Acquisition with FTK Imager
\item Memory Analysis – physical memory / RAM. A LOT can be obtained from RAM, like passwords. Analyzing memory dumps
\item Strings – ‘old skool’: Sysinternals’ strings – defaults to Unicode and ASCII, minimum length 3 characters – a lot of interesting info is not in a printable format.
\item Volatility & YARA: Volatility – advanced memory forensics framework, python, plugins, free tools, malware detection tools. YARA – malware plugins for volatility, easy to write custom extensions
\item Lab 2- Memory Analysis with volatility – analyzing a sample memory dump with volatility. 
\end{itemize}

\newline
Advanced Forensics Lesson 2
\begin{itemize}
\item Core Windows Forensics
\item Investigation Methodology from Day 1
\item Virtually everything done in Windows refers to or is recorded into the Registry
\item RegMon (sysinternals) can be used to display registry activity in real time. 
\item Registry access barely remains idle.
\item By opening the Registry Editor (regedit), the Registry can be seen as one unified “file system”. Five most hierarchal folders are called “hives” and being with HKEY. Only two real HKU and HKLM. The other three are shortcuts or alias to branches to withint one of the two hives
\item Each of the five hives is composed of keys, which contain values and subkeys
\item The registry as a forensic log
\item Timeine creation and analysis: once you obtain info you could check file access, creation and modify times around the period – you can get some idea of the actions taken place and correlate that with other time stamped files
\item Lab 4: Creating Timeline of MFT
\item List of things to look for: relevant files, windows event logs, application config files and logs, prefetch folder. 
\item In many cases, the target system is a VM.
\item File and directory analysis
\item Data recovery – recovers files or data deleted or in slack from the system
\item Lab 4 – Recovering Data (with PhotoRec)
\item Case info and team work lab
\end{itemize}

% --------------------------------------------------------------
%     You don't have to mess with anything below this line.
% --------------------------------------------------------------
 
\end{document}
