% --------------------------------------------------------------
% This is all preamble stuff that you don't have to worry about.
% Head down to where it says "Start here"
% --------------------------------------------------------------
 
\documentclass[12pt]{article}
\usepackage{graphicx}
\graphicspath{ {images/} }
 
\usepackage[margin=1in]{geometry} 
\usepackage{amsmath,amsthm,amssymb}
\setlength{\parskip}{1em}

\newcommand{\N}{\mathbb{N}}
\newcommand{\Z}{\mathbb{Z}}
 
\newenvironment{theorem}[2][Theorem]{\begin{trivlist}
\item [\hskip \labelsep {\bfseries #1}\hskip \labelsep {\bfseries #2.}]}{\end{trivlist}}
\newenvironment{lemma}[2][Lemma]{\begin{trivlist}
\item [\hskip \labelsep {\bfseries #1}\hskip \labelsep {\bfseries #2.}]}{\end{trivlist}}
\newenvironment{exercise}[2][Exercise]{\begin{trivlist}
\item [\hskip \labelsep {\bfseries #1}\hskip \labelsep {\bfseries #2.}]}{\end{trivlist}}
\newenvironment{problem}[2][Problem]{\begin{trivlist}
\item [\hskip \labelsep {\bfseries #1}\hskip \labelsep {\bfseries #2.}]}{\end{trivlist}}
\newenvironment{question}[2][Question]{\begin{trivlist}
\item [\hskip \labelsep {\bfseries #1}\hskip \labelsep {\bfseries #2.}]}{\end{trivlist}}
\newenvironment{corollary}[2][Corollary]{\begin{trivlist}
\item [\hskip \labelsep {\bfseries #1}\hskip \labelsep {\bfseries #2.}]}{\end{trivlist}}

\newenvironment{solution}{\begin{proof}[Solution]}{\end{proof}}
 
\begin{document}
 
% --------------------------------------------------------------
%                         Start here
% --------------------------------------------------------------
 
\title{Week 8 Writeup}
\author{Arthur Liou}

\maketitle

Prompt: Submitting a write-up of your thoughts, impressions, and any conclusions based on the material from the week. Each week will have its own assignment in the grades page.
\par

\linebreak
For the first part of this week’s writeup, I’m reflecting on the topic – Messaging Security. I really loved this week. Not as many slides, but also there was a lot of content and lab work in this week, which I really like for learning because I could play around with the materials and content without having any expectations (group work/hw type) set upon what I’m learning. Some of the terminology and tools I knew and had used before, but there were much more that I didn’t know before. It was great learning more about this from a educational cybersecurity standpoint rather than a personal and professional viewpoint. All in all, an excellent week of material!
For what we covered and learned, see my lecture notes below.

\newpage
Lecture Notes: Messaging Security 
\newline
Lesson 1 – Messaging Security
\begin{itemize}
\item Phishing Quiz
\item Terminology - Spam/Ham, Strap/Honeypot, Botnet, Snowshoe spam, Phishing vs Spear Phishing, RBL, Heuristics, Bayesian (Statistical), Fingerprinting/Hashing
\item History and Evolution of Spam, botnets
\item Technology to combat spam: Engines
\item Reputation-driven: IP, message, URL
\item Content-driven: common strings, fixed strings vs variable strings (regular expression), message attributes, combo of strings and attributes
\item Tools for Messaging Data – Linux tools, open-source DBs, regex coach, trustedsource.org, spamhaus.org
\item Tools for research purposes – Dig (Domain Information Groper, WHOIS
\item Demo 1 – Postgres – Exploration.
\item Demo 2 – Regex Coach + Group Practice
\item Research Techniques for managing the data flood: Samples metadata: Parsing, Grouping, Aggregation, ID of outliers
\item Considerations – Human input required, fully automated, combination of auto and human input?, probability scoring vs additive scoring
\item Lab 1 – Data Exploration. Total, distinct, average, types of files and their extensions
\item Lab 1 – Representative Delegation
\item Nominate an individual to present the group’s analysis and findings at the end of Lab 2
\item Recap; Key Concepts – Data Model – Spam/Ham

Lesson 2 – Lecture Wrap and Classification Lab
\item SMTP Conversation – Ham, Spam, Email Header Reading
\item Data Model – Spam, Ham
\item The Data Scientific Method
\item 1. Start with data.
\item 2. Develop intuitions about the data and the questions it can answer.
\item 3. Formulate your question.
\item 4.Leverage your current data to better understand if it is the right
\item question to ask. If not, iterate until you have a testable hypothesis.
\item 5. Create a framework where you can run tests/experiments.
\item 6. Analyze the results to draw insights about the question.
\item Classification Lab - The provided message data table has 100k rows of real-world message meta data. Use the tools and techniques covered to make spam/ham decisions for all records
\item SQL Examples. Useful Operators:
\item COUNT()
\item DISTINCT()
\item SPLITPART()
\item GROUP BY col
\item ORDER BY col
\item Spam is pervasive - Digital & Printed media, Audio/Visual
\item Many aspects of Security can be reduced to finding the least common denominator among large data sets
\item Automate “Finding the needle”
\item Classification accuracy is directly tied to the depth in which we are able to describe samples
\item Tools
\item Spamhaus RBL
\item McAfee RBL
\item The Regex Coach
\item Trustedsource.org
\item Domaintools.net
\item Reputationauthority.org
\item Yougetsignal.com/tools/web-sites-on-web-server/
\item Spamassassin.apache.org
\item PostgreSQL

\end{itemize}
% --------------------------------------------------------------
%     You don't have to mess with anything below this line.
% --------------------------------------------------------------
 
\end{document}
